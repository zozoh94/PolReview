\documentclass{article}
\usepackage[utf8]{inputenc}
\usepackage{empheq}

\title{TAL - Politic review}
\author{ Ismail Erradi, Björn Goriatcheff, Enzo Hamelin  }
\date{06 May 2017}

\begin{document}

\maketitle

\section{Introduction}
    In this project, analyzing language syntax and semantic is used in order to review, rate and classify ideas extracted from the program of every presidential candidates. On 5 different topics, we analyzed what was the opinion of each candidate and how those ideas were correlate.
    
\section{Goals}
\begin {itemize}
\item Extracting/parsing opinion of candidates on 5 different subjects.
\item Weighting them using appropriate metric.
\item Emphasizing the emergence of a group.
\item Implementing an ask-answer system.
\end{itemize}


\section{Functions}
\subsection{Data preparation}
\paragraph{Dictionary}
The listing of every candidates have been implemented in a dictionary to create aliases.
The listing of subjects have also been implemented in a dictionary containing the synonyms and other words associated to the subject.
Both dictionaries are included in the file "settings.py"
\paragraph{Programs}
Every presidential program have been previously downloaded and imported into the folder "projects".

\subsection{Parsing}
In order to parse the programs, we used textract library to parse pdf encoded files out to text.
Additionally, this text was tokenized using nltk library.

\subsection{Weighting}
Weighting correctly generated tokens was the most difficult part of our project. In order to do so, we semantically analyze a sentence of tokens. If this sentence contained a keyword of our dictionary then the proposition was weighted according to the grammar (affirmative sentence or negative sentence).
Every similar proposition is weighted the same way and counted as a positive or negative proposition. 
\subsection{Classify}
To classify the position of a candidate on a precise subject we count how many times this subject was recall in the program and how was the recall (positive or negative) compared to the original opinion. A simple calculus is executed to know if we can trust the opinion using the following formula: 
\begin{empheq}{equation*}
\large trust-for = \frac {positive}{positive+negative+neutral}
\end{empheq}
\begin{empheq}{equation*}
\large trust-against = \frac {negative}{positive+negative+neutral}
\end{empheq}
\section{Results}
Analyse des programmes...
\paragraph{Libre échange :\\}
Nathalie Arthaud

Phrases concernées : 0 
Avis pour : 0
Avis contre : 0
\\
Philippe Poutou

Phrases concernées : 3
Avis pour : 0
Avis contre : 1
\\
Jean-Luc Mélenchon

Phrases concernées : 7
Avis pour : 0
Avis contre : 11
\\
Jacques Cheminade

Phrases concernées : 4
Avis pour : 1
Avis contre : 2
\\
Benoît Hamon

Phrases concernées : 6
Avis pour : 2
Avis contre : 1
\\
Emmanuel Macron

Phrases concernées : 0
Avis pour : 0
Avis contre : 0
\\
Jean Lassalle

Phrases concernées : 1
Avis pour : 0
Avis contre : 0
\\
François Fillon

Phrases conceres : 2
Avis pour : 0
Avis contre : 1
\\
Nicolas Dupont-Aignan


Phrases concernées : 13
Avis pour : 0
Avis contre : 4
\\
François Asselineau


Phrases concernées : 2
Avis pour : 0
Avis contre : 0
\\
Marine Le Pen


Phrases concernées : 3
Avis pour : 0
Avis contre : 2



\section{To Achieve}
\begin{itemize}
    \item Addition of more subjects (Social, Society, Constitutional organization, State organization, Employment, Firms, Start-up...)
    \item Deeper analyze of sentence, recognition of complex sentences, metaphor..
    \item More calculus in order to have precise results
\end{itemize}


\end{document}

